\section{Presheaves as a Topos}

\begin{exercise}{9.2}
    Let $\Ccal$ be a category with pullbacks. 
    \begin{exercises}
        \item Show that $\Sub(X)$ is a meet semi-lattice for each object $X$ in $\Ccal$.
        \item Show that if $f\colon Y\to X$ is a morphism in $\Ccal$, then $f^*\colon\Sub(X)\to\Sub(Y)$ is a morphism of meet semi-lattices.
        \item Show that we have a presheaf
            \[ \Sub\colon\Ccal^\op\to\Set.\]
    \end{exercises}
\end{exercise}
\begin{solution}
    \begin{exercises}
        \item\begin{proof}
            Consider two monomorphisms $A\xrightarrow{n}X\xleftarrow{m}b$. 
            Then we can define $n\wedge m$ to be any composition along the pullback square
            \begin{tikzcd}
                C \arrow[d] \arrow[r] & B \arrow[d, "m"] \\
                A \arrow[r, "n"]      & X               
            \end{tikzcd}, and note that pullbacks of monos are monos (\ref{ex:25}). It is straighforward to verify that this defines a meet operation. 
            If $\Ccal$ has an initial object, then $\Sub(X)$ has a bottom element. 
        \end{proof}
        \item\begin{proof}
            For $n\colon A\to X$ in $\Sub(X)$, define $f^*(n)$ to be the pullback of $n$ along $f$, i.e.
            \begin{tikzcd}
                C \arrow[d, "f^*(n)"'] \arrow[r] & A \arrow[d, "n"] \\
                Y \arrow[r, "f"]                 & X               
            \end{tikzcd}. 
        \end{proof}
        \item\begin{proof}
            To show that $\Sub(fg)=\Sub(g)\Sub(f)$ note that we have the commutative diagram
            \begin{center}
                \begin{tikzcd}
                    C \arrow[d, "(fg)^*(n)"'] \arrow[r] & B \arrow[d, "f^*(n)"] \arrow[r] & A \arrow[d, "n"] \\
                    Z \arrow[r, "g"']                   & Y \arrow[r, "f"']               & X               
                \end{tikzcd}
            \end{center}
            where the right and composite squares are pullbacks, and by \ref{ex:26} it follows that the left square is a pullback, i.e. $(fg)^*(n)=g^*(f^*(n))$.
            To show that $\Sub(\id_X)=\id_{\Sub(X)}$ observe that 
            \begin{tikzcd}
                A \arrow[d, "n"] \arrow[r, "\id_A"] & A \arrow[d, "n"] \\
                X \arrow[r, "\id_X"']               & X               
            \end{tikzcd}
            is a pullback square. 
        \end{proof}
    \end{exercises}
\end{solution}

\begin{exercise}{9.3}
    Show that in $\Set$ we have for each set $X$ an isomorphism of posets:
    \[ \Sub(X)\cong(\Pcal(X),\subseteq). \]
\end{exercise}
\begin{solution}
    \begin{proof}
        The isomorphism is given by sending each monomorphisms $n\colon A\to X$ in $\Sub(X)$ to $\im n$ and each subset $A\subseteq X$ to the inclusion map $i_A\colon A\to X$. 
        This maps are morphisms of posets since for $m\leq n$ we have $m(A)=nk(A)$ so $\im m\subseteq\im n$ and if $A\subseteq B$ then we have the commuting diagram 
        \begin{tikzcd}
            A \arrow[d, "i_A"] \arrow[r, "i"] & B \arrow[ld, "i_B"] \\
            X                                 &                    
            \end{tikzcd}
            of inclusion maps, so $i_A\leq i_B$.
    \end{proof}
\end{solution}

\begin{exercise}{60}
    Let $(P,\leq)$ be a preorder. For $p\in P$ we let $\downarrow p =\{q\in P\mid q\leq p\}$.
    Show that sieves on $p$ can identified with downward closed subsets of $\downarrow p$.
    If we denote the unique arrow $q\to p$ by $qp$ and $U$ is a downwards closed subset of $\downarrow p$, what is $(qp)^*U$?
\end{exercise}
\begin{solution}
    \begin{proof}
        
    \end{proof}
\end{solution}

\begin{exercise}{61}
    Let $\Ccal$ be a small category.
    Show that the power object in the category of presheaves on $\Ccal$ can be identified as follows $\Pcal(X)(C)=\Sub(X\times y_C)$ and that for $f\colon C'\to C$, $\Pcal(X)(f)(U)=(\id_X\times y_f)^*(U)$.
    Prove also that the element relation, as a subpresheaf $\in_X$ of $\Pcal(X)\times X$ is given by
    \[ (\in_X)(C)=\{(U,x)\in\Sub(y_C\times X)\times X(C)\mid(\id_C,x)\in U(C)\}. \]
\end{exercise}
\begin{solution}
    \begin{proof}
        
    \end{proof}
\end{solution}

\begin{exercise}{62}
    
\end{exercise}
\begin{solution}
    \begin{proof}
        
    \end{proof}
\end{solution}

\begin{exercise}{63}
    
\end{exercise}
\begin{solution}
    \begin{proof}
        
    \end{proof}
\end{solution}